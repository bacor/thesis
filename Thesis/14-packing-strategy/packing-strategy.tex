\documentclass{../src/bcthesispart}
\title{Reformulating the packing strategy}
\author{Bas Cornelissen}
\begin{document}

%——————————————————————————————————————————————————————————

\appendixtitle{Reformulating the packing strategy}%
	{Reformulating the packing strategy}%
	{packing-strategy}{%
	%
	% Abstract
	% ————————
	The technical formulation of the packing strategy in \parencite{Hurford1975} seems to have caused some confusion in the literature.
	This appendix reformulates the principle independent of the original generative framework, without compromising preciseness.
	This will bring some limitations of the packing strategy to the fore.
}

%——————————————————————————————————————————————————————————


%——————————————————————————————————————————————————————————
\paragraph{The packing strategy as a constraint on trees}

The packing strategy was introduced within the conceptual framework of generative grammar, as a ‘significant generalisations’ about number expressions and how they relate to numbers.
\textcite{Hurford1975} analysed several numeral systems (English, French, Danish, Mixtec and Yoruba) using a phrase structure grammar which can be simplified to:\footnote{%
	%>>>
	The original phrase structure rules constructed bases using exponentiation. 
	This is controversial (see chapter \ref{ch:numerals}) so I have use the most recent, simplified grammar from \textcite{Hurford2007}.
	Note that the rewrite rule of \PROD{} is different in \textcite{Hurford1987}, where \SUM{} is not optional.
	I have also changed notation and use \SUM{} for \textsc{num}; \PROD{} for \textsc{phrase}; and \ATOM{} for \textsc{digit}.
	%<<<
	}
%-
\begin{align}
	\label{eq:app-ps:hurford-grammar}
	%-----
	\begin{split}
		\text{\SUM} 
			&\longrightarrow \begin{Bmatrix*}[l]
				\text{\ATOM}\\
				\text{\PROD{} (\SUM)}
			\end{Bmatrix*}
		\\
		\text{\PROD} 
			&\longrightarrow \;\text{(\SUM) \BASE}
	\end{split}
\end{align}
%-
where \ATOM{} and \BASE{} rewrite to one of the atoms and bases of the system respectively.
It is easiest to think of this grammar as an \emph{attribute grammar} \parencite{Knuth1968} where every leaf (\ATOM{} or \BASE{}) has a fixed numeric value or \emph{attribute}. 
Every nonterminal node corresponds to an operation that computes the value of the node from the values of its constituents.
\SUM{}s of course correspond to sums and \PROD{}s to products.
Here is the structure for French \lng{quatre-vingt-dix-sept}, where I decorated nodes with their attributes in grey:
%-
\begin{align}
	\label{eq:app-ps:tree-quatre-vingt-dix-sept}
	%-----
	\begin{tikzpicture}[frontier/.style={distance from root=10em}]
		\Tree 
		[.\SUM{}\attr{97} 
			[.\PROD{}\attr{80} 
				[.\SUM\attr{4} [.\ATOM{}\attr{4} quatre ] ]
				[.\BASE{}\attr{20} vingt ] ]
			[.\SUM{}\attr{17} 
				[.\PROD{}\attr{10} 
					[.\BASE{}\attr{10} dix ] ]
				[.\SUM{}\attr{7} 
					[.\ATOM{}\attr{7} sept ] ] ] ]
	\end{tikzpicture}
\end{align}
%-
%%




This is just one of the many structures with value 97 generated by the rules \eqref{eq:app-ps:hurford-grammar}. 
The packing strategy was introduced as a way to separate the wellformed from the illformed structures.
It was therefore formulated as constraint on the structure of the trees, namely that:\footnote{%
	%>>>
	The formulation is from \textcite{Hurford1987,Hurford2007}. 
	The original also applied to bases constructed by exponentiation and is thus more complicated, as \BASE{} nodes were nonterminals.
	Let $A$ be a structure of category $X$ (i.e.\ a \PROD{} 	or a \BASE{}) with value $x$ and two constituents: a \SUM{} and some node of another category $Z$ (\PROD{} or \BASE{}). 
	Then $A$ is only wellformed if $Z$ has the largest possible value $z \le x$. 
	That is, if there is no alternative $Z'$ that also expands $X$ with $\text{val}(Z) < \text{val}(Z')$.
	%<<<
	}
%-
\begin{align}
	\label{eq:app-ps:packing-strategy}
	%-----
	\text{the sister constituent of a \SUM{} must have the highest possible value.}
\end{align}
%-
That is, the highest possible value while keeping the value of the parent constant.
The sister constituent of a \SUM{} can be a \PROD{} or a \BASE{}. 
Both can be found in \eqref{eq:app-ps:tree-quatre-vingt-dix-sept}: at depth 3 we for example find a \BASE{} with value 20 and a \PROD{} with value 10.
The reader might also have noticed that the node \SUM{}\attr{17} \emph{violates} the packing strategy. 
In a structure of the form
%-
\begin{align}
	\label{eq:app-ps:counterexample-98}
	%-----
	\begin{tikzpicture}
		\Tree 
		[.\SUM{}\attr{97} 
			\PROD{}\attr{90} 
			\SUM{}\attr{7} ]
		\end{tikzpicture}
\end{align}
%-
the node \PROD{}\attr{90} is the sister constituent of a \SUM{} and has a value higher than 80.
We will discuss this problem later in more detail.
%%




The packing strategy also accounts for the order of bases in large numerals, e.g.\ that that \lng{two hundred thousand} is wellformed, but \lng{two thousand hundred} is not:
%-
\begin{align}
	\label{eq:counterexample-98}
	%-----
	\begin{tikzpicture}[frontier/.style={distance from root=15em}]
		\Tree 
		[.\SUM{}\attr{200\;000} 
			[.\PROD{}\attr{200\;000}
				[.\SUM{}\attr{200} 
				 	[.\PROD{}\attr{200} 
				 		[.\SUM{}\attr{2} 
				 			[.\ATOM{}\attr{2} two ] ]
				 		[.\BASE{}\attr{100} hundred ] ] ]	
				[.\BASE{}\attr{1000} thousand ] ] ]
	\end{tikzpicture}
	%
	\hspace{3cm}
	%
	\begin{tikzpicture}[frontier/.style={distance from root=15em}]
		\Tree 
		[.\SUM{}\attr{200\;000} 
			[.\PROD{}\attr{200\;000}
				[.\SUM{}\attr{2000} 
				 	[.\PROD{}\attr{2000} 
				 		[.\SUM{}\attr{2} 
				 			[.\ATOM{}\attr{2} two ] ]
				 		[.\BASE{}\attr{1000} thousand ] ] ]	
				[.\BASE{}\attr{100} hundred ] ] ]
	\end{tikzpicture}
\end{align}
%-
In the tree on the right, the sister node of \SUM{}\attr{2000} is the node \BASE{}\attr{100}, and this violates the packing strategy, as it is also possible to form a tree where the corresponding sister has the higher value 1000. 
This is the tree shown on the left.
%%




%——————————————————————————————————————————————————————————
\paragraph{The packing strategy without trees}

Perhaps the tree representations overly complicated.\footnote{%
	%>>>
	I doubt whether Hurford would disagree; over the years he used ever looser variants of the grammar, and often opts for arithmetic formulae in the discussion \parencite{Hurford1999,Hurford2007}.
	
	%<<<
	}
First they generate obvious redundancies in fragments like
%-
\begin{align}
	\label{eq:chains}
	%-----
	\Tree [.\PROD{}\attr{10} [.\BASE{}\attr{10} dix ] ]
	\qquad\qquad\text{and}\qquad\qquad
	\Tree [.\SUM\attr{4} [.\ATOM{}\attr{4} quatre ] ]
\end{align}
%-	
But more importantly, the same structures can be expressed using simple arithmetic formulae like $(4 \times 20) + (10 + 7)$ and $(2 \times 100) \times 1000$, as we have been doing throughout. More precisely, every tree corresponds to a formula built up from the values of the leaves using the binary operations addition and multiplication.
Such formulae are not only simpler, they also have more expressive power. 
The phrase structure rule can for example only produce multiplicative constructions with a base. 
But as we have seen, languages sometimes contain multiplicative constructions with factors that are \emph{not} considered bases properly: isolated or mixed bases.
Similarly, additive constructions with two atoms are illformed by the packings strategy; a base has to figure in one of the constituents.
The Welsh expression for 15 is an additive base, but not a base. 
This means that \emph{correct} expressions of the form $15+2$ cannot be generated.\footnote{%
	%>>>
	\textcite{Hurford1975} does list 15 as a base, and thus circumvents this at the cost of using an arguably wrong notion of base.
	%<<<
	}
It is furthermore easy to extend the formulae with other (binary) operations like subtraction and division, whereas the phrase structure rules can only account for these using complicated extensions of the semantic interpretation which I will not reproduce here.
Finally, the order of constituents, of the base and atom in particular, cannot be described in the formalism, which is problematic \parencite{Calude2016}. 
In short, the formulae are simpler, more expressive and stay closer to the semantic structure of number expressions.





So how can we express the packing strategy in terms of such formulae?
Well, note that \SUM{}s only occur as sister constituents (bold) in fragments of the form
%-
\begin{align*}
	%(A)
	\text{(a)}
	\Tree [.\SUM{} 
			[.\PROD{} \SUM{} \BASE{} ] 
			\textbf{\SUM{}} ]
	%(B)
	\qquad\quad\text{(b)}
	\Tree [.\SUM{} 
			[.\PROD{} \BASE{} ] 
			\textbf{\SUM{}} ]
	%(C)
	\qquad\quad\text{(c)}
	\Tree [.\PROD{} \textbf{\SUM{}} \BASE{} ]
\end{align*}
%-
which, when collapsing chains, correspond to formulae of the form 
\begin{align*}
	\text{(a)}\;\; (y \times b) + x,\qquad
	\text{(b)}\;\; b + x, \qquad
	\text{(c)}\;\; x \times b.
\end{align*}
Here $x$ is the sum of interest, $y$ some other expresion and $b$ a base. 
Sister constituents of \SUM{}s are thus multiples of bases (considering $b = 1\times b$ a multiple) and the packing strategy states that these should have the highest possible value.
We can thus reformulate the packing strategy as:
%-
\begin{align}
	\label{eq:new-packing-strategy}
	%-----
	\text{Complex numerals use the largest multiple of the largest base possible.}
\end{align}
%-
This directly suggests more general principles, such as:
%-
\begin{align}
	\label{eq:generalized-packing-strategy}
	%-----
	\text{The difference between $a$ and $b$ in $a+b$ and $a\times b$ should be maximised.}
\end{align}
%-
This would also apply to multiplicative constructions like $5 \times 6$, which do not contain a base (in English).
The principle would then correctly favour $3 \times 10$.
Principle (\ref{eq:generalized-packing-strategy}) could be taken as a good interpretation of the informal statement that “languages prefer to form numeral expressions by combining constituents whose arithmetical values are maximally apart, within the constraints defined by the syntax of the system” \parencite[243]{Hurford1987}.
But this is \emph{not} a literal reformulation of the packing strategy: it is slightly more general.
%%




%——————————————————————————————————————————————————————————
\paragraph{Limits of the packing strategy}

One of the arguments for the importance of packing strategy was that it explained the peculiarities of French numerals \textcite{Hurford1975}.
As a final note, I would like to point out that Hurford’s explanation is somewhat problematic.
Recall that structure \eqref{eq:app-ps:counterexample-98} showed that French numerals do not satisfy the packing strategy. 
So how does Hurford use the packing strategy to explain why \lng{soixante dix} ($60 + 10$) is wellformed, and $50+20$ or $40+30$ are not? 
Consider the following expressions for 70:
%-
\begin{align*}
	\begin{tabular}{lll}
		(a) $7 \times 10$				
			& \textbf{(b) $6 \times 10 + 10$}
			& (c) $5 \times 10 + 20$\\
		(d) $3 \times 20 + 10$
			& (e) $2 \times 20 + 3 \times 10$
			&
	\end{tabular}
\end{align*}
%-
The correct expression is (b), although some dialects use \lng{septante} for (a).
We can directly eliminate (c) since the packing strategy favours $(6 \times 10)+10$ over $(5\times 10)+20$.
But (b) is illformed, since  $6 \times 10$ is illformed in the light of $3 \times 20$ (it is assumed that 20 is a base).
To correct for this, two additional constraints are introduced \parencite[101]{Hurford1975}.
The first states (in a complicated way) that $70 \times 10, 80 \times 10$ and $90\times 10$ are illformed. This eliminates (a). 
The second states (in an even more complicated way) that all multiples or 20, except $4 \times 20$, are illformed. 
This eliminates (d) and (e), but also makes (b) \emph{well}formed, as desired.




The Packing Strategy, in short, does not appear to \emph{explain} much about the French numerals.
On the contrary, it would predict a quite different, vigesimal system, which can only be remedied by introducing ad-hoc constraints.
This is perhaps not surprising. 
The packing strategy predicts a completely regular numeral system, and it is hard to see how such a strategy in itself could account for irregularities like those encountered with French numerals.
These conclusions do mean that the packing strategy might not be the important generalisation Hurford suggests it to be.
The corresponding generalisations in \textcite{Greenberg1978} (roughly, 37 and 38) might even be of more empirical relevance.
The latter captures that numeral systems are very predictable, or that “there is no ‘surprise’ in numeral larger than [a certain] base”. 
If the French expression for 70 is irregular, so is the expression for 70 in $170 = 100+70$.
Finally, to the best of my knowledge the packing strategy has never been \emph{systematically} evaluated against a large collection of numeral systems either.
This might be something to address in future work, which might benefit from the simplified formulation of the generalised packing strategy derived in this appendix.


\end{document}
